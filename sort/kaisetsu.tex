\documentclass[slide,20pt]{ltjsarticle}

\usepackage{_my_style}

\usepackage{listings}
\usepackage{xcolor}
\usepackage{tikz}
\usepackage{diagbox}
\usepackage{fancyhdr}

\lstset{
  basicstyle=\ttfamily,
  keywordstyle=\color{blue},
  commentstyle=\color{green},
  stringstyle=\color{red},
  showstringspaces=false,
  language=C++,
}

% フッターにページ番号を表示するカスタムコマンド
\makeatletter
\def\@oddfoot{
  \hfil \normalsize\rmfamily\thepage\hfil % ページ番号を中央に配置
}
\def\@evenfoot{\@oddfoot} % 偶数ページにも同じフッターを適用
\makeatother

\renewcommand{\lstlistingname}{リスト}

\begin{document}

% 目次ページの前にカスタムヘッダー/フッターを設定
\pagestyle{fancy}
\fancyhf{} % 既存のヘッダーとフッターをクリア
\fancyfoot[C]{} % フッターの中央を空白にする
\renewcommand{\headrulewidth}{0pt} % ヘッダーの罫線を消す

% 目次
\tableofcontents
\newpage

\pagestyle{plain}

\setcounter{page}{2}

\section{はじめに}

コードは全部\textcolor{red}{\href{https://github.com/kk2a/algo_data}{ここ}}にあるので,コードは直接見てください.

考える配列の長さは特に断りがない限り,$N$とする.

\section{バブルソート}

\begin{itemize}
  \item 「隣り合う要素を比較して,順番が逆であれば交換する」というのを「操作」と呼ぶことにする.バブルソートは,「操作」を繰り返すことで,数列を昇順に並べ替えるアルゴリズムである.
  \item $i$回の「操作」を行うと,数列の右端から少なくとも$i$個の要素が正しい順番に並ぶ.$1$回の「操作」で$O(N)$回のアクセスを行い,ソート全体で「操作」が行われる回数は高々$N$回なので,時間計算量は$O(N ^ 2)$である.
  \item ソートが完了していたら,直ちに処理を終了するような実装をする場合,$1$回の「操作」で終われば,時間計算量は$O(N)$でこれが最良である.
  \item 逆に,完全に逆順になっている数列に対してバブルソートを行うと,$N$回の「操作」を行うことになり,時間計算量は$O(N ^ 2)$となる.
\end{itemize}

\section{バケットソート}

\begin{itemize}
  \item 最大値を$R$, 最小値を$L$とする.$R - L + 1$の大きさの配列をもって,各値が何回出現したかを記録して,その情報をもとにソートする.
  \item $M = R - L + 1$と置けば,時間計算量は$O(M + N)$, 空間計算量は$O(M)$である.
  \item $M$がかなり大きいと空間がすごく無駄になるので,座標圧縮したくなるが,結局座標圧縮でソートしないといけないので本末転倒である.
\end{itemize}

\section{ヒープソート}

\begin{itemize}
  \item 二分ヒープを用いたソートアルゴリズムである.
  \item 根の取得$O(1)$と,根の更新$O(\log N)$が$N$回行われるので,時間計算量は$O(N \log N)$である.
\end{itemize}

\subsection{二分ヒープ}

一応,二分ヒープの説明をする.ヒープソートはin-placeにやるので,0-indexedの二分ヒープを使う.ノードの添え字は,図\ref{fig:heap}のようにつけ,それぞれのノードに値は値を持っている.

\begin{figure}[h]
  \centering
  \begin{tikzpicture}[
    level distance=1.5cm,
    level 1/.style={sibling distance=6cm},
    level 2/.style={sibling distance=3cm},
    level 3/.style={sibling distance=1.5cm},
    every node/.style={circle, draw, minimum size=0.75cm},
  ]
  
    % Root node
    \node {0}
      % Level 1
      child {node {1}
        % Level 2
        child {node {3}
          % Level 3
          child {node {7}}
          child {node {8}}
        }
        child {node {4}
          % Level 3
          child {node {9}}
        }
      }
      child {node {2}
        % Level 2
        child {node {5}}
        child {node {6}}
      };
  
  \end{tikzpicture}
  \caption{二分ヒープ}
  \label{fig:heap}
\end{figure}

親ノードの値は,子ノードの値以上になるようにする.構築は,葉から根に向かって,違反があればswapするというのを繰り返せばよく,根の更新は,根から葉に向かって,違反があればswapするというのを繰り返せばよい.最初の構築は$O(N)$,再構築は高さが$O(\log N)$なので$O(\log N)$である.

\section{挿入ソート}

\begin{itemize}
  \item 「$i$番目から$0$番目までこの順番で見ていって,逆の順番があれば交換する」というのを「$i$番目の操作」と呼ぶことにする.挿入ソートは,$i = 1, \cdots N - 1$に対して,この順番で$i$番目の操作を行うことで,数列を昇順に並べ替えるアルゴリズムである.
  \item $j$番目の操作までを行ったとき,$j$番目まではソート済みだから,$j + 1$番目の操作では,正しい順番になった瞬間$j + 1$番目の操作を終了することができる.そのため,すでにソートされている場合,$i$番目の操作は$O(1)$だから,時間計算量は$O(N)$でこれが最良である.
  \item 逆に,完全に逆順になっている数列に対して挿入ソートを行うと,$i$番目の操作では,$O(N)$回のアクセスが行われるため,時間計算量は$O(N ^ 2)$となる.
\end{itemize}

\section{マージソート}

\begin{itemize}
  \item マージソートは,分割統治法を用いたアルゴリズムである.
  \item $N > 2$個の要素を$N / 2$個の要素に分割し,それぞれを再帰的にソートし,$2$つのソート済みの配列の大きくないほうを順番に取り続けるという操作で,$N$個の要素をソートする.$N \le 2$の時は,普通にする.
  \item マージソートの時間計算量を$T(N)$とおくと,$T(N) = 2T(N / 2) + O(N)$なので,$T(N) = O(N \log N)$である ($O(N)$の$O(\log N)$個の和になっている.) .
\end{itemize}

\section{クイックソート}

\begin{itemize}
  \item クイックソートは,マージソートとだいたい同じだが分割の仕方が違う.
  \item ピボット呼ばれる値を$1$つ選び,それ未満とそれ以上に分割する.ピボットの選び方が悪いと,$1$個と$N - 1$個の分割が繰り返されることになって,時間計算量が$O(N ^ 2)$になる.
  \item ここでは,$0$番目の値をピボットとして選ぶことにしているので,完全に逆順になっていたり,最初からソート済みであったりしたら,時間計算量は$O(N ^ 2)$である.
\end{itemize}

\section{時間計測}

\begin{itemize}
  \item ここでは,$N \le 10 ^ 6$のときの時間計測を行う.
  \item 表\ref{tab:time_random}は,$0$から$N - 1$までの整数をランダムに割り当てたときの時間である.
  \item 表\ref{tab:time_worst_b}, \ref{tab:time_worst_q}は,最悪の場合の時間である.
\end{itemize}

\begin{table}[h]
  \centering
  \caption{ランダムな数列の時間 (単位: ms)}
  \label{tab:time_random}
  \begin{tabular}{|c|c|c|c|c|c|c|c|c|c|c|}
    \hline
    $N$ & バブル & バケット & ヒープ & 挿入 & マージ & クイック \\\hline
    $10 ^ 3$ & 1 & 1 & 1 & 1 & 1 & 1 \\\hline
    $10 ^ 4$ & 185 & 1 & 1 & 7 & 1 & 1 \\\hline
    $10 ^ 5$ & 114176 & 5 & 50 & 41169 & 68 & 51 \\\hline
    $10 ^ 6$ & \diagbox{}{}  & 5 & 65 & \diagbox{}{}  & 82 & 51 \\\hline
  \end{tabular}
\end{table}

\begin{table}[h]
  \centering
  \begin{minipage}{0.45\textwidth}
    \centering
    \caption{最悪の場合の時間 (バケット) (単位: ms)}
    \label{tab:time_worst_b}
    \begin{tabular}{|c|c|}
      \hline
      $M$ & バケット \\\hline
      $10 ^ 7$ & 61 \\\hline
      $10 ^ 8$ & 421 \\\hline
      $10 ^ 9$ & 4198 \\\hline
    \end{tabular}
  \end{minipage}
  \hfill
  \begin{minipage}{0.45\textwidth}
    \centering
    \caption{最悪の場合の時間 (挿入,クイック) (単位: ms)}
    \label{tab:time_worst_q}
    \begin{tabular}{|c|c|c|}
      \hline
      $N$ & 挿入 & クイック \\\hline
      $10 ^ 3$ & 1 & 1 \\\hline
      $10 ^ 4$ & 800 & 391 \\\hline
      $10 ^ 5$ & 80812 & Error \\\hline
    \end{tabular}
  \end{minipage}
\end{table}

\section{おまけ}

一般に全順序が与えられたときでもソートができる.比較が$O(1)$なら,同様に$O(N \log N)$でできる.例えば,逆順にソートしたいときとか,前半に偶数,後半に奇数を入れるみたいな特殊なソートもできる.詳しくは,最初に示したリンクから見てください.

\end{document}