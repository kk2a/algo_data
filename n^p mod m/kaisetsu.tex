\documentclass[slide,20pt]{ltjsarticle}

\usepackage{_my_style}

\usepackage{listings}
\usepackage{xcolor}

\lstset{
  basicstyle=\ttfamily,
  keywordstyle=\color{blue},
  commentstyle=\color{green},
  stringstyle=\color{red},
  showstringspaces=false,
  language=C++,
}

% フッターにページ番号を表示するカスタムコマンド
\makeatletter
\def\@oddfoot{
  \hfil \normalsize\rmfamily\thepage\hfil % ページ番号を中央に配置
}
\def\@evenfoot{\@oddfoot} % 偶数ページにも同じフッターを適用
\makeatother

\renewcommand{\lstlistingname}{リスト}

\begin{document}

\setcounter{page}{2}
\section*{問題}
整数$N, M, P$が与えられる.このとき,$N ^ P$を$M$で割った余りを求めよ.

\subsection*{制約}
\begin{description}
\item[・] $1 \leq N, M \leq 10^9$
\item[・] $1 \le P \le 10^{14}$
\end{description}

\subsection*{出典}
\href{https://atcoder.jp/contests/atc002/tasks/atc002_b}{AtCoder Typical Contest 002 B}


\section*{コード解説}
\subsection*{normal1.cpp}
for文を用いて$ans$に$N$を$P$回かける.計算量は$O(P)$.コードはリスト$1$に示す.

\lstinputlisting[caption={normal1.cpp}]{normal1.cpp}

\newpage
\subsection*{normal2.cpp}
再帰関数を用いた累乗のナイーブな実装.

\[ N ^ X = N ^ {X - 1} \cdot N \]

を用いて再帰的に計算する.計算量は$O(P)$.コードはリスト$2$に示す.

\lstinputlisting[caption={normal2.cpp}]{normal2.cpp}

\newpage

\subsection*{kuhuu1.cpp}
繰り返し$2$乗法の再帰帰数を用いた実装.

\begin{equation*}
  N ^ X = \left\{
    \begin{array}{ll}
      N ^ {X / 2} \cdot N ^ {X / 2} & (X \text{が偶数}) \\
      N ^ {(X - 1) / 2} \cdot N ^ {(X - 1) / 2} \cdot N & (X \text{が奇数})
    \end{array}
  \right.
\end{equation*}

を用いて再帰的に計算する.計算量は$O(\log P)$.コードはリスト$3$に示す.

\newpage
\lstinputlisting[caption={kuhuu1.cpp}]{kuhuu1.cpp}

\newpage

\subsection*{kuhuu2.cpp}
繰り返し$2$乗法の非再帰的な実装.

\[ X = \sum_{i = 0} ^ {h} a_i \cdot 2 ^ i, \]

のように書けたとき,

\[ N ^ X = \left(N ^ {2 ^ 0}\right) ^ {a_0} \cdot \left(N ^ {2 ^ 1}\right) ^ {a_1} \cdot \cdots \cdot \left(N ^ {2 ^ h}\right) ^ {a_h} \]

という性質 (分配法則です.) を使い,$N ^ {2 ^ i}$を計算しながら,$a_i = 1$の時に,$ans$に$N ^ {2 ^ i}$をかける.

\[ N ^ {2 ^ {i + 1}} = N ^ {2 ^ {i}} \cdot N ^ {2 ^ {i}} \]

であるから,$N ^ {2 ^ i}$は簡単に計算できる.$h = \lceil \log_2(X) \rceil$だから,計算量は$O(\log P)$.コードはリスト$4$に示す.

\newpage
\lstinputlisting[caption={kuhuu2.cpp}]{kuhuu2.cpp}

\section*{計測}
実行時間の測定は,AtCoderのコードテストで行った.ローカル環境だと,PCの個体差が出てしまうと考えたためである.言語はC++ 23 (gcc 12.2)を用いている.また,見やすくするために,normal2とkuhuu1のコードを比較し (どちらも再帰.) ,計算量に強く依存している$P$以外の$N, M$は固定する.ここでは,$N = 3, M = 100$とした.計測時間の比較を表$1$に示す.

\begin{center}
\begin{table}[ht]
  \centering
  \caption{実行時間の比較}
\begin{tabular}{|c|c|c|c|}  
  \hline
  $P$ & normal (ms) & kuhuu (ms) & normal/kuhuu   \\
  \hline
  $10 ^ 1$ & 1 & 1 & 1 \\
  \hline
  $10 ^ 2$ & 1 & 1 & 1 \\
  \hline
  $10 ^ 3$ & 1 & 1 & 1 \\
  \hline
  $10 ^ 4$ & 1 & 1 & 1 \\
  \hline 
  $10 ^ 5$ & 2 & 1 & 2 \\
  \hline
  $10 ^ 6$ & 11 & 1 & 11 \\
  \hline
  $10 ^ 7$ & 108 & 1 & 108 \\
  \hline 
  $8 \cdot 10 ^ 7$ & 851 & 1 & 851 \\
  \hline
\end{tabular}
\end{table}
\end{center}
normalは$P = 9 \cdot 10 ^ 7$でスタックオーバフローでエラーを吐いた.$P = 10 ^ 5$程度までならば,normalとkuhuuの実行時間はほぼ変わらない.しかし,$P = 10 ^ 7$程度までになると,その比は$100$倍以上になっている.kuhuuは$P = 10 ^ {18}$まで計測したが,ずっと$1 \mathrm{ms}$であった.

\section*{まとめ}
繰り返し$2$乗法の実装は,再帰関数を用いたものと,用いなかったもののどちらも,Github Copilotに大まかに実装してもらい,少し修正を加えている (余りを取るところ.) .

\begingroup
\let\clearpage\relax
\section*{参考記事}
\endgroup
\begin{itemize}
\item \href{https://qiita.com/drken/items/3b4fdf0a78e7a138cd9a#4-%E7%B4%AF%E4%B9%97-an}{「998244353 で割ったあまり」の求め方を総特集! 〜 逆元から離散対数まで 〜 4章 累乗}
\end{itemize}

\end{document}